% Параметры компиляции

\documentclass[a4document]{article}

\usepackage[english,russian]{babel}
\usepackage[unicode, pdftex]{hyperref}




\thispagestyle{empty} 

\usepackage[14pt]{extsizes}
\usepackage[left=2cm,right=2cm,
    top=2cm,bottom=2cm]{geometry}
\usepackage{graphicx}
\usepackage{subcaption}

\begin{document}

% Титульный лист
{
\centering{
Министерство образования и науки Российской Федерации \\
Федеральное государственное автономное образовательное \\
учреждение высшего профессионального образования \\
«Национальный исследовательский Нижегородский государственный университет им. Н.И. Лобачевского»\\
Институт информационных технологий, математики и механики \\
\bigbreak
\bigbreak
\bigbreak
Кафедра Математического обеспечения и суперкомпьютерных технологий \\
Направление подготовки "Программная инженерия"
\bigbreak
\bigbreak
\bigbreak
\bigbreak
\bigbreak
\bigbreak
\bigbreak
\textbf{Отчет по лабораторной работе №2 \\
"Кликер"}



\bigbreak
\bigbreak
}

\begin{flushright}
    \textbf{Выполнил: } \\
    студент группы 382008-1 \\
    Булгаков Д.Э. \\
    \bigbreak
    \textbf{Руководитель: }  \\ 
    доцент кафедры программной \\
    инженерии \\
    Борисов Н.А.
    
\end{flushright}

\vspace*{\fill}

\begin{center}
Нижний Новгород \\
2023 г.
\end{center}
}



% Оглавление
{
\newpage
\begin{flushleft}
\tableofcontents
\end{flushleft}
}

% -0-
{
\newpage
\section{Цель.}
\par 
Научиться использовать основные объекты qml.

}

% -0-
{
\newpage
\section{Постановка задачи.} 
\par 
\begin{enumerate}
    \item Добавить объект кнопки.
    \item Добавить текстовое поле с счетчиком. 
    \item При нажатии на кнопку значение в текстовом поле увеличивается на единицу.
    \item Собрать на эмуляторе и протестировать.
\end{enumerate}

}

% -0-
{
\newpage
\section{Руководство пользователя.} 
\par 
Начальный экран приложения выглядит следующим образом (рис. 1)
\begin{figure}[htp]
    \centering
    \includegraphics[width=0.5\linewidth]{resources/1.png}
    \caption{Начальный экран.}
    \label{fig:pk1}
\end{figure} 

\par \noindent
При нажатии на кнопку значение в текстовом поле увеличивается на единицу (рис. 2)
\begin{figure}[htp]
    \centering
    \includegraphics[width=0.5\linewidth]{resources/2.png}
    \caption{После нажатия.}
    \label{fig:pk2}
\end{figure}

}

% -0-
{
\newpage
\section{Приложение.} 
\par {MainPage.qml}
\begin{verbatim} 
import QtQuick 2.0
import Sailfish.Silica 1.0

Page {
    objectName: "mainPage"
    allowedOrientations: Orientation.All
    PageHeader {
        id : pheader
        objectName: "pageHeader"
        title: qsTr("ЛАБА")

    }
    Column {
        id : clmn
        width: parent.width

        anchors.top : pheader.bottom
        property int counter: 0

        Label {
            anchors.horizontalCenter: parent.horizontalCenter
            text: clmn.counter
        }

        Button {
            anchors.horizontalCenter: parent.horizontalCenter
            text: "нАжМи МеНя"
            onClicked: clmn.counter++
        }
    }
}
\end{verbatim}

}


\end{document}